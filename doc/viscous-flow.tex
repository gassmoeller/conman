\documentclass[12pt,twoside]{article}
\usepackage{times}
\begin{document}


\title{EAS 591D Mantle Dynamics and Thermodyanmics}
\author{Scott D. King}
\date{Spring 2006}
\maketitle

\newcommand{\dx}{\delta x}
\newcommand{\dz}{\delta z}
\newcommand{\uxz}{u \left ( x , z \right )}
\newcommand{\Uxz}{\vec u \left ( x , z \right )}
\newcommand{\vxz}{v \left ( x , z \right )}
\newcommand{\uxpz}{u \left ( x + \dx , z \right )}
\newcommand{\vxzp}{v \left ( x , z + \dz \right )}
\newcommand{\rxz}{\rho \left ( x , z \right )}
\newcommand{\rxpz}{\rho \left ( x + \dx , z \right )}
\newcommand{\rxzp}{\rho \left ( x , z + \dz \right )}
\newcommand{\drdt}{{{\partial \rxz}\over{\partial t}}}
\newcommand{\dudx}{{{\partial \uxz}\over{\partial x}}}
\newcommand{\dvdz}{{{\partial \vxz}\over{\partial z}}}
\newcommand{\drdx}{{{\partial \rxz}\over{\partial x}}}
\newcommand{\drdz}{{{\partial \rxz}\over{\partial z}}}

\section{Viscous Flow}
\pagestyle{myheadings}
\markboth{EAS 591d}{Mantle Dynamics and Thermodynamics}

This course is designed to give the student an understanding of the
basic equations for viscous flow and their solution on a computer.   I assume
that the student is reasonably familiar with vector calculus and linear algebra
and is at least not afraid of partial differential equations.  In addition, it
will be very helpful to have a general familiarity and experience with Fortran
programming.  Even though the goal of this course is to have the student solving
viscous flow problems on the computer, we will work out a number of analytic
solutions.  It is critical to have a working knowledge of simple analytic
solutions, both to test one's computer code against and to build up one's
intuition of viscous flow problems.  There will be no exams for this class and
most all of the assignments will have a computer aspect.   The class will be very
informal.  Once we cover the basics, there may be a lot of discussion of the
results of assignments in class.  In some cases the student should be forewarned
that not all of the answers to the assignments will be known before hand. The
course will concentrate on the finite element method, since that is what I
typically use in my research.   However, if time and interests permit, we will
look at some of the recent advances in high order finite differences and
spectral methods.

Test for commit

\begin{quotation} 
\noindent
First Note:  I will adopt an {\em Eulerian} frame of reference
in this course.  That means that we will consider our points of observation
fixed and watch the fluid as it passes by them.   This is in contrast to a {\em
Lagrangian} framework (formulation), where instead of a fixed set of observation
points, we track the path of a fixed set of fluid parcels.  Both viewpoints have
advantages and drawbacks.   For most convection problems, tracking the paths of
a set of fluid parcels quickly becomes an intractable problem.  In some cases,
we truly want to know the origin of a parcel of fluid.   Later in the course we
will discuss some methods for tracking parcels of fluid in an Eulerian system.
\end{quotation}
 
\begin{quotation} 
\noindent
Second Note:  Unless otherwise stated, I will be using a 2D
Cartesian geometry.  First, because it will make working problems on the
blackboard easier and illustrates all of the basic principles (where 3D becomes
important  I will discuss it).  Second, the basic form of ConMan is 2D Cartesian.
\end{quotation}

\subsection{Derive Equations}

We begin with two basic physical principals; conservation of mass and
conservation of momentum (this is truly a poor choice of name for the second
principal, and we will see why in the following sections).  There are a number
of ways to derive the basic equations and several forms that we can write these
equations in (e.g., stress and strain-rate or velocities).  Because I come from
a fluid dynamics background and because of the formulation used in ConMan, I
will begin with the velocity formulation; however, I will do many of the
derivations several ways because often different problems can be solved much
more easily by using one form or the other.   A familiarity with both forms is
useful.  

This material is covered in a number of textbooks on fluid dynamics.  There are
several which I find quite useful:

\begin{quotation}
\noindent
{\em Physical Fluid Dynamics,} D. J. Tritton, Oxford University Press, 1988.
{\em The equations of motion are discussed in Chapter 5. Chapter 8 will also be
quite useful for us.  This is pretty much at the level of my notes.}
\medskip

\noindent
{\em Fluid Mechanics,} L. D. Landau and E. M. Lifshitz, Pergamon Press, 1959.
{\em The equations of motion are derived on the first few pages, actually
Chapters 1 and 2 are quite useful.}
\medskip

\noindent
{\em An Introduction to Fluid Dynamics,} G. K. Batchelor, Cambridge, 1967.
{\em Chapter 3 is somewhat useful, but quickly gets into rotation.  Batchelor
is a classic fluids text.}
\medskip

\noindent
{\em Geodynamics: Applications of Continuum Physics to Geological Problems,} D.
L. Turcotte and G. Schubert, Wiley, 1982. {\em Chapter 6 covers fluid mechanics
at a similar level to this class - they only consider viscous flow.}
\medskip

\noindent
{\em Hydrodynamic and Hydromagnetic Stability,} S. Chandrasekhar, Oxford,
1961. {\em This book is THE classic text for convection.  The level of
mathematical expertise required is high, but you should at least know of its
existence.}
\end{quotation}
 
\subsubsection{Conservation of Mass} 

Let us begin by considering an infinitesimal box whose sides are of length $\dx$
by $\dz$.   We consider the mass flowing into and out of the box in a time
$\delta t$.

\begin{picture}(300,250)(-40,0)
\put(75,25){\framebox(150,150){}}
\put(45,100){\line(1,0){25}}
\put(230,100){\line(1,0){25}}
\put(150,0){\line(0,1){25}}
\put(150,180){\line(0,1){25}}
\put(35,110){$\uxz$}
\put(235,110){$\uxpz$}
\put(105,10){$\vxz$}
\put(85,190){$\vxzp$}
\put(10,85){\line(0,-1){60}}
\put(5,100){$\dz$}
\put(10,115){\line(0,1){60}}
\put(135,240){\line(-1,0){60}}
\put(150,235){$\dx$}
\put(165,240){\line(1,0){60}}
\end{picture}

\begin{eqnarray}
\drdt \dx \dz & = & \left [ \uxz \rxz - \uxpz \rxpz \right ]\dz \nonumber \\ 
&  &  + \left [ \vxz \rxz - \vxzp \rxzp \right ]\dx \label{eq:cont1}
\end{eqnarray}
now I use the approximations
\begin{eqnarray}
\uxpz & = & \uxz + (\dx)\dudx \label{eq:uxpz} \\
\vxzp & = & \vxz + (\dz)\dvdz \label{eq:vxzp}\\
\rxpz & = & \rxz + (\dx)\drdx \label{eq:rxpz}\\
\rxzp & = & \rxz + (\dz)\drdz \label{eq:rxzp}.
\end{eqnarray}
Using \ref{eq:uxpz}, \ref{eq:vxzp}, \ref{eq:rxpz}, and \ref{eq:rxzp}
for the second and fourth terms on the right hand side of  \ref{eq:cont1}, we get
\begin{eqnarray}
& \uxpz \rxpz & =  \nonumber \\
&\left (\uxz+(\dx)\dudx\right )\left (\rxz+(\dx)\drdx\right ) & = \nonumber
\\        
& \uxz \rxz + \uxz \drdx \dx + \rxz \dudx \dx & +  \nonumber \\ 
& \dudx \drdx (\dx)^2 
\end{eqnarray}
and
\begin{eqnarray} 
& \vxzp \rxpz & =  \nonumber \\ 
&\left (\vxz+(\dz)\dvdz\right )\left (\rxz+(\dz)\drdz\right ) & = \nonumber
\\         
& \vxz \rxz + \vxz \drdz \dz + \rxz \dvdz \dz & +  \nonumber \\  &
\dvdz \drdz (\dz)^2 
\end{eqnarray}
Dropping terms of order $(\dx)^2$ and $(\dz)^2 $ and simplifying,
equation~\ref{eq:cont1} becomes
\begin{eqnarray}
\drdt & = & - \left [ \uxz \drdx + \rxz \dudx \right ] \nonumber \\  
&  &      - \left [ \vxz \drdz + \rxz \dvdz \right ]
\label{eq:cont2}
\end{eqnarray}
or in vector notation
\begin{eqnarray}
\drdt & = & - \rxz \nabla \cdot \Uxz - \Uxz \cdot \nabla \rxz \nonumber \\
& = &  - \nabla \cdot \left ( \rxz\Uxz \right )
\label{eq:veccont2}
\end{eqnarray}
where
\begin{equation}
\Uxz = \left ( \uxz , \vxz \right ).
\end{equation}

Note a special case of interest is when the density $\rho$ is constant. In this
case, equation~\ref{eq:veccont2} reduces to
\begin{equation}
\nabla \cdot \left ( \Uxz \right ) = 0.
\label{eq:incompress-cont}
\end{equation}

A slightly cleaner version of the conservation of mass can be derived if we
stick to vector notation throughout the derivation.  Because I will present the
momentum equation fully in vector notation, it is worth going through the
derivation again:

\begin{picture}(300,100)(-40,-20)
\put(150,25){\circle{40}}
\put(155,30){\line(1,1){20}}
\put(155,35){$\vec u$}
\put(150,40){\line(0,1){20}}
\put(130,45){$\vec dS$}
\end{picture}

We will consider some arbitrary volume of fluid, $V$, with a surface,  $S$.
We will define $d\vec S$, a vector with an outward pointing normal to the surface
$S$. Then, the rate of loss of mass from $V$ will equal
\begin{equation}
\oint \rho \vec u \cdot d\vec S.
\label{eq:rml}
\end{equation}
Transforming equation~\ref{eq:rml} into an integral over volume,
\begin{equation}
\int \nabla \cdot (\rho \vec u ) dV.
\label{eq:rml2}
\end{equation}
The total mass in volume $V$ is given by
\begin{equation}
\int \rho dV,
\end{equation}
hence,
\begin{equation}
{d\over dt}\int \rho dV = -\int \nabla \cdot (\rho \vec u ) dV.
\end{equation}
Because we consider a constant volume
\begin{equation}
\int  {{\partial \rho}\over{\partial t}} dV = -\int \nabla \cdot (\rho \vec
u ) dV.
\label{eq:cont3}
\end{equation}
The only way equation~\ref{eq:cont3} can be valid over the volume $V$ is if at
every point
\begin{equation}
{{\partial \rho}\over{\partial t}} + \nabla \cdot (\rho \vec u ) = 0.
\label{eq:cont4}
\end{equation}
Which should look familiar.

\subsubsection{Conservation of Momentum}

The next conservation equation, often referred to as the conservation of
momentum, is not so easy to derive.  I am also going to drop the $(x,z)$
notation from my variables, but remember that in general velocities and
densities can be a function of their location in space and time.

We will consider some arbitrary volume of fluid, $V$, with a surface,  $S$,
as in the picture above.  The total force, $T_f$ acting on this volume of fluid
is equal to the integral of the pressure, $p$, taken over the surface of the
fluid.
\begin{equation}
- \oint p dS = T_f
\label{eq:pres1}
\end{equation}
Transforming \ref{eq:pres1} to a volume integral
\begin{equation} 
- \oint p dS =  - \int \nabla p dV.
\label{eq:pres2}
\end{equation}

At this point, I am going to neglect possible effects due to dissipation of
energy of the fluid.  We will build that in at the next step.  From
equation~\ref{eq:pres2}, we see that the fluid surrounding any volume element
$dV$ exerts a force on that element of $- \nabla p$. That force must be balanced by a change in acceleration
of the fluid according to Newton's second law.   
\begin{equation}
{{D (\rho\vec u)} \over {D t}} = - \nabla p
\label{eq:euler1}
\end{equation}
Notice that 
\begin{equation} 
{{D (\rho\vec u)}\over{D t}}=\rho{{D\vec u}\over{Dt}} + \vec u {{D \rho}
\over {D t}}
\label{eq:totalmomentum}
\end{equation}
however the second term in Equation~\ref{eq:totalmomentum} is zero from
Equation~\ref{eq:cont4}  The only physical mechanisms for this would be changes
in phase or if velocities approached the speed of light and both of these cases
would (and can) be handled as special cases.   However, in most problems of
geophysical interest, the fluids are not moving at anything approaching the
speed of light.   A fluid with a change in phase could possibly be a special
topic to be covered later.

Note that ${{D \vec u} \over {D t}}$ is not the rate of change of the fluid
velocity at a fixed point in space, but the rate of change of a given fluid
particle as it moves about in space (that is why I did not write it in the
usual way).   However, because we are in an Eulerian framework, it has to be
expressed in terms of quantities referring to a fixed point in space.   To
accomplish this, consider an arbitrarily small parcel of fluid moving between two
points (arbitrarily close), denoted by the vector $d \vec r$. The parcel has two
components to its change in velocity $d\vec u$, namely the change due to the
change during the time $\delta t$ at a point fixed in space, and the
difference between the velocities, at the same instant in time, at the two
points $d \vec r$ apart.   We can write this as
\begin{equation}
d \vec u = \left ({{\partial \vec u} \over {\partial t}} \right ) \delta t +
(d\vec r \cdot \nabla) \vec u,
\end{equation}
or dividing both sides by $\delta t$,
\begin{equation} 
{{D \vec u} \over {D t}} = {{\partial \vec u} \over {\partial t}} + 
(\vec u \cdot \nabla) \vec u.
\label{eq:total-deriv}
\end{equation}
Substituting \ref{eq:total-deriv} into \ref{eq:euler1}, we get
\begin{equation}  
{{\partial \vec u} \over {\partial t}} + (\vec u \cdot \nabla) \vec u =
- {1\over\rho}\nabla p.
\label{eq:euler-final}
\end{equation}
Equation~\ref{eq:euler-final} is the equation of motion of a fluid, with no
dissipation.   It was first derived by L. Euler in 1755 and is known as {\em
Euler's equation}.   If the fluid is in a gravitational field we need to add an
additional body force to the right hand side $\rho \vec g$ where $\vec g$ is
the acceleration due to gravity.  This modifies \ref{eq:euler-final}, 
\begin{equation}   
{{\partial \vec u} \over {\partial t}} + ( \vec u \cdot \nabla ) \vec u = 
- {1\over\rho}\nabla p + \vec g.
\label{eq:euler-final2}
\end{equation}
This is the basic equation governing fluid motion.  It is the fundamental
equation for atmospheric science, hydrology, and most other fluid sciences,
even though the way some of these fields write their basic equations many
appear in a different form than \ref{eq:euler-final2}.

This is a good place to mention the {\em Boussinesq Approximation,} or what is
sometimes called {\em incompressible flow.}  In the Boussinesq approximation, we
assume that the density is constant everywhere, except allowing the density to
vary in the buoyancy term of equation~\ref{eq:euler-final2}.  Thus, the
continuity equation reduces to
\ref{eq:incompress-cont} and Euler's equation becomes

\begin{equation}    
{{\partial \vec u} \over {\partial t}} + ( \vec u \cdot
\nabla ) \vec u =  - {1\over\rho_o}\nabla p + {\rho\over\rho_o}\vec g.
\label{eq:euler-final3}
\end{equation}
You might wonder about the validity of this approximation.   However, you can
convince yourself that if $||{{D \vec u} \over {D t}}||$ is very small compared
to $||\vec g||$, as it most certainly is in most geologically relevant cases,
then the error made by neglecting changes in density in this term will be
small. The term ${\rho\over\rho_o}\vec g$ in equation~\ref{eq:euler-final3} is
the connection between density perturbations from temperature, which drive flow
in the thermal convection problem.   Without variations in density, the flow is
only driven by pressure variations and/or velocity boundary conditions.

ConMan, and many other fluid flow codes use the Boussinesq approximation;
however, a number of researchers have shown that in mantle convection
compressibility effects (changes in density) can have important effects on the
flow, so compressible flow is becoming more and more common.   If time permits,
we will examine compressibility in greater detail later in the course.

\subsubsection{Conservation of Momentum: Viscous Dissipation}

We now need to add the viscous forces acting on the element.  For this we need
to recall some basic principals of stress and strain-rate.  If you need to
refresh your memory Turcotte and Schubert Chapter 2 is a pretty good place to
look.   If you look in Tritton or Landau and Lifshitz, you will be pretty
disappointed, probably because most fluid dynamics problems are not dominated
by stress and strain-rate.  

It is probably worth a brief digression to remind everyone of the fundamentals
of stress and strain (strain-rate).   If a solid is {\em elastic,} then
stress is linearly related to strain.   If a material is a {\em Newtonian
fluid,} then stress is linearly related to strain-rate.  I find it easiest to
think in terms of elastic solids and then take the time derivative to get
strain-rates.  I can't think of any particular reason why this is not a valid
thing to do. 

Lets consider a 2D box, whose dimensions are $\dx$ and $\dz$.  Lets squeeze the
box so that it's new dimensions are $\dx^\prime$ $\dz^\prime$, we could imagine
doing this by increasing hydrostatic pressure.   Notice that in general the box
may also translate, i.e., the center of mass moves.  This translation, should
not effect the strain in any way.   In addition, there could be rigid body
rotation, this also should not effect the strain.

\begin{picture}(300,250)(-40,0)
\put(75,25){\framebox(150,150){}}
\put(115,55){\framebox(130,130){}}

\put(10,85){\line(0,-1){60}}
\put(5,100){$\dz$}
\put(10,115){\line(0,1){60}}
\put(135,240){\line(-1,0){60}}
\put(150,235){$\dx$}
\put(165,240){\line(1,0){60}}
\put(30,105){\line(0,-1){50}}
\put(25,120){$\dz^\prime$}
\put(30,135){\line(0,1){50}}
\put(165,220){\line(-1,0){50}}
\put(180,215){$\dx^\prime$}
\put(195,220){\line(1,0){50}}
\end{picture}

Now I should be careful in my use of notation, because we are doing an elastic
problem, so velocities become displacements, to remind you of that, I'm going to
use $w$ for displacement; however, lets charge merrily along...   Lets assume
the lower left corner of  the box is at $(x,z)$ and after the ``strain event''
is at
$(x^\prime,z^\prime)$.  Then the {\em displacement} of the lower left corner
will be
\begin{eqnarray}
w_x (x,z) = x - x^\prime \\
w_z (x,z) = z - z^\prime 
\end{eqnarray}
and for the lower right hand corner will be
\begin{equation} 
w_x (x + \dx,z) = (x + \dx) - (x^\prime + \dx^\prime ) 
\end{equation}
and for the upper left hand corner
\begin{equation}
w_z (x ,z + \dz) = (z + \dz) - (z^\prime + \dz^\prime ).
\end{equation}
Now, we once again do our usual trick, using a first-order Taylor expansion for
the terms $w_x (x + \dx,z)$ and $w_z (x ,z + \dz)$, since $\dx$ and $\dz$ are
infinitesimal.
\begin{eqnarray}
w_x (x + \dx,z) & = & w_x (x,z) + {{\partial w_x }\over{\partial x}} \dx \\
w_z (x ,z + \dz) & = & w_x (x,z) + {{\partial w_z }\over{\partial z}} \dz 
\end{eqnarray}
by substituting and rearranging, we can get
\begin{eqnarray}
\dx & = & \dx^\prime + {{\partial w_x }\over{\partial x}} \dx \\
\dz & = & \dz^\prime + {{\partial w_z }\over{\partial z}} \dz.
\end{eqnarray}
Thus, if we define strain as the change in length divided by the original
length, we get
\begin{eqnarray}
\epsilon_{xx} & = & {{\dx -\dx^\prime}\over \dx} = {{\partial w_x
}\over{\partial x}}
\\
\epsilon_{zz} & = & {{\dz -\dz^\prime}\over \dz} = {{\partial w_z }\over{\partial
z}}.
\end{eqnarray}

This simple little analysis only dealt with normal strains, that is shortening
(or lengthening) in the direction parallel to the strain.  We now need to look
at shear strains.   For that we need the figure below.

\begin{picture}(300,250)(-40,0)
\put(25,25){\framebox(150,150){}}
\put(25,25){\line(4,1){150}}
\put(25,25){\line(1,4){37.5}}
\put(175,62.5){\line(1,4){37.5}}
\put(62.5,175){\line(4,1){150}}
\put(10,85){\line(0,-1){60}}
\put(5,100){$\dz$}
\put(10,115){\line(0,1){60}}
\put(85,240){\line(-1,0){60}}
\put(100,235){$\dx$}
\put(115,240){\line(1,0){60}}
\end{picture}

I will define the shear strain as one-half the decrease in the right angle in
the lower left corner.   If I define $\phi_1$ and $\phi_2$ as the angles
through which the sides of the original quadrilateral are rotated,
then 
\begin{equation}
\epsilon_{xz} = -{1\over2} ( \phi_1 + \phi_2 )
\end{equation}
Notice that by this definition, the shear strain is symmetric, i.e.,
\begin{equation}
\epsilon_{xz} = \epsilon_{zx}.
\end{equation}
Further, we can relate  $\phi_1$ and $\phi_2$ to the displacements and use the
fact that the displacements are infinitesimal, so that we can use $\tan \alpha
= \alpha $
\begin{eqnarray}
\tan ~ \phi_1 & = & {{-w_z (x + \dx, z)}\over \dx} = \phi_1 \\
\tan ~ \phi_2 & = & {{-w_x (x , z + \dz )}\over \dz} = \phi_2
\end{eqnarray}
Once again, we will use our first-order Taylor expansion trick again.
Jumping straight to the result, the relation between shear strain and the
derivatives of displacement
\begin{equation}
\epsilon_{xz} = {1\over 2} \left ( {{\partial w_z}\over {\partial x}} +
{{\partial w_x}\over {\partial z}} \right ).
\end{equation}

For completeness, shear strain can lead to a solid body rotation.  To see this,
we need the definition of the solid body rotation about the $y$ axis
\begin{equation}
\omega_{y} = -{1\over2} ( \phi_1 - \phi_2 ) = {1\over 2} \left ( {{w_z}\over
\dx} - {{w_x}\over \dz} \right )
\end{equation}
Using the relations above, you can write
\begin{eqnarray}
\phi_1 = - (\epsilon_{xz} + \omega_y ) \\
\phi_2 = - (\epsilon_{xz} - \omega_y )
\end{eqnarray}
For the case $\phi_1 = \phi_2$, we have pure shear, and $\omega_y = 0$. For the
case where one of the angles is zero, we have both a shear strain and a
rotation.

To convince yourself that you understand what is going on, it would be
wise to look at a few simple problems.
\begin{quote}
{\bf Problem 1}  Add two simple shear problems together, one in the $x$
direction and one in the $z$ direction, both with the same angle $\alpha$. 
Show the resulting shear stress and rotation.
\end{quote}

\begin{quote}
{\bf Problem 2} Show for the triangle below that
\begin{equation}
\tau_{x^\prime x^\prime} = \tau_{xx} \cos^2 \theta +  \tau_{zz} \sin^2 \theta +
\tau_{xz} \sin 2\theta
\end{equation}
(Note, I drew it like $\theta = \pi/4$ but you can not assume this!)
\end{quote}

\begin{picture}(300,180)(-40,0)
\put(125,25){\line(0,1){150}}
\put(125,25){\line(1,0){150}}
\put(125,175){\line(1,-1){150}}
\put(75,100){\line(1,0){20}}
\put(80,105){$\tau_{xx}$}
\put(200,0){\line(0,1){20}}
\put(190,10){$\tau_{zz}$}
\put(200,100){\line(1,1){20}}
\put(210,110){$\tau_{x^\prime x^\prime}$}
\put(105,80){\line(0,1){40}}
\put(110,85){$\tau_{xz}$}
\end{picture}

\vfill\eject

And now that our brief tour of strain is over, we will go back to fluids.  In
everything that follows, I will try and stick to the following convention,
deviatoric stress will be denoted by $\tau$, deviatoric strain-rate by $\dot
\epsilon$ and total stress by $\sigma$.

\newcommand{\txxxz}{\tau_{xx} \left ( x , z \right )}
\newcommand{\txxxpz}{\tau_{xx} \left ( x +\delta x , z \right )}
\newcommand{\tzzxz}{\tau_{zz} \left ( x , z \right )}
\newcommand{\tzzxpz}{\tau_{zz} \left ( x + \dx , z \right )}
\newcommand{\tzzxzp}{\tau_{zz} \left ( x , z + \dz \right )}
\newcommand{\txzxz}{\tau_{xz} \left ( x , z \right )}
\newcommand{\txzxpz}{\tau_{xz} \left ( x + dx , z \right )}
\newcommand{\tzxxz}{\tau_{zx} \left ( x , z \right )}
\newcommand{\tzxxzp}{\tau_{zx} \left ( x , z + \dz \right )}

\begin{picture}(300,250)(-40,-20)
\put(75,25){\framebox(150,150){}}
\put(40,100){\line(1,0){25}}
\put(235,100){\line(1,0){25}}
\put(150,-10){\line(0,1){25}}
\put(150,190){\line(0,1){25}}
\put(30,110){$\txxxz$}
\put(240,110){$\txxxpz$}
\put(95,10){$\tzzxz$}
\put(75,190){$\tzzxzp$}

\put(10,85){\line(0,-1){60}}
\put(5,100){$\dz$}
\put(10,115){\line(0,1){60}}
\put(135,240){\line(-1,0){60}}
\put(150,235){$\dx$}
\put(165,240){\line(1,0){60}}
\end{picture}

The first thing to notice is that if there is going to be no net torque,
then
\begin{equation}
\txzxz = \tzxxz.
\end{equation}
Next, following what we did in the case of conservation of mass, 
the net viscous force in the x direction per unit area is
\begin{equation}
{{\left [ \txxxpz - \txxxz \right ] \dz}\over \dx\dz} + 
{{\left [ \tzxxzp - \tzxxz \right ] \dx}\over \dx\dz} = 0.
\label{eq:xxstress}
\end{equation}
Similarly for the $z$ direction, we have
\begin{equation} 
{{\left [ \tzzxpz - \tzzxz \right ] \dx}\over \dx\dz} + 
{{\left [ \txzxpz - \txzxz \right ] \dz}\over \dx\dz} = 0.
\label{eq:zzstress}
\end{equation}

Once again we do our Taylor expansion, for the $\txxxpz$ type terms and
equation~\ref{eq:xxstress} reduces to
\begin{equation}
{{\partial\txxxz}\over{\partial x}} + {{\partial\tzxxz}\over{\partial z}} =0
\label{eq:xxstress2}
\end{equation}
and equation~\ref{eq:zzstress} reduces to
\begin{equation} 
{{\partial\tzzxz}\over{\partial z}} + {{\partial\txzxz}\over{\partial x}} =0.
\label{eq:zzstress2}
\end{equation}

We can write equations~\ref{eq:xxstress2} and~\ref{eq:zzstress2} in a general
tensor notation as
\begin{equation}  
\tau_{ij,i} = 0
\end{equation}
where repeated subscripts mean to sum the terms over, in this case $i=1,2$ and
$j=1,2$.

For an ideal {\em Newtonian} viscous fluid, viscous stresses are linearly
proportional to velocity gradients (by definition).   This leads us to
\begin{eqnarray}
\tau_{xx} & = & 2 \mu {{\partial u}\over{\partial x}} \\
\tau_{zz} & = & 2 \mu {{\partial v}\over{\partial z}} \\
\tau_{xz} & = & \tau_{zx} = \mu \left (  {{\partial u}\over{\partial z}} + 
{{\partial v}\over{\partial x}} \right )
\end{eqnarray}
where $\mu$ is the dynamic viscosity (Pa s).  Recall that a Pascal is ${\rm
kg} / {\rm m s^2}$ is more fundamental units.    Notice that in
$ij$ notation we could write
\begin{equation}
\tau_{ij} = \mu \dot\epsilon_{ij}
\end{equation}
Substituting for velocity derivatives in equation~\ref{eq:xxstress2}, we get
\begin{equation}
{ {\partial \left (2 \mu {{\partial u}\over{\partial x}} \right ) } \over
{\partial x} } + { {\partial \left ( \mu \left (  {{\partial u}\over{\partial z}}
+  {{\partial v}\over{\partial x}} \right ) \right )} \over {\partial z} } = 0
\end{equation}
and a similar messy equation for~\ref{eq:zzstress2}.   Notice that {\em if and
only if} the viscosity is a constant, we can simplify the above to
\begin{equation}  
2 \mu {{\partial^2 u}\over{\partial x^2}}   + 
\mu \left (  {{\partial^2 u}\over{\partial z^2 }} + 
{{\partial^2 v}\over{\partial x \partial z}} \right ) = 0.
\end{equation}
Now we return to the conservation of mass equation~\ref{eq:incompress-cont} for
the incompressible case, and differentiate each side by $\partial x$
\begin{equation}
{{\partial^2 v}\over{\partial x \partial z}} = - {{\partial^2 u}\over{\partial
x^2 }}
\end{equation}
so in this case, we arrive at 
\begin{equation}
\mu \left (  {{\partial^2 u}\over{\partial x^2 }} +  {{\partial^2
u}\over{\partial z^2}} \right ) = 0.
\end{equation}
In a exactly the same way, we can get the term for the viscous stress in the
$z$ direction
\begin{equation}
\mu \left (  {{\partial^2 v}\over{\partial x^2 }} +  {{\partial^2
v}\over{\partial z^2}} \right ) = 0.
\end{equation}
or in vector notation
\begin{equation}
\mu \nabla^2 \vec u = 0
\end{equation}
Notice in vector notation, we can write the case where $\mu$ is not constant
more cleanly,
\begin{equation}
\nabla ( \mu \nabla \vec u) = 0.
\end{equation}

If we add this term to our Euler equation, from back on
equation~\ref{eq:euler-final2}, we have
\begin{equation}    
{{\partial \vec u} \over {\partial t}} + ( \vec u \cdot
\nabla ) \vec u =  - {1\over\rho}\nabla p + \vec g + {1\over\rho}
\nabla ( \mu \nabla \vec u).
\label{eq:navier-stokes1}
\end{equation}

Now we have the most general equation for describing a the motion of an
irrotational, non-magnetic fluid.   If you want to see the effects of rotation
and magnetic fields, go look at Chandrasekhar.  Sometimes the term $\mu/\rho$ is
called the dynamic viscosity, with units of ${\rm m^2}/ s$.   Thus, special care
needs to be taken when discussing viscosity.

\vfill\eject

\subsection{Non-Dimensionalization}

What is non-dimensionalization?   We rewrite the equations in terms of
dimensionless variables by dividing out a constant characteristic size for the
length, time, velocity, etc in the equations.  Because these are constant, we
can take them outside the derivatives and group them into parameters.   These
parameters allow us to see the relationship between the variables and to scale
problems from the laboratory or computer, to Earth scales.  

Why do we want to non-dimensionalize equations?   Often,
the equations we want to solve in fluid problems can not be solved with all the
terms present,  sometimes terms that make an equation intractable do not
contribute to the physics of the problem we want to solve.   Also, by using
dimensional analysis, we can often gain insight into what physical processes
control (dominate) the problems we are interested in.   As we will see later,
this is also something important to do on a computer where we have a finite
precision.

It is helpful to think of our variables in terms of the fundamental units of
mass, $M$, length, $L$, and time, $T$
\begin{eqnarray*}
x & \rightarrow & L \\
t & \rightarrow & T \\
\rho & \rightarrow & {M\over L^3} \\
\rho \vec g & \rightarrow & {{M}\over{L^2 T^2}} \\
\tau & \rightarrow & {M\over {LT^2}} \\
p & \rightarrow & {M\over {LT^2}} \\
\dot \epsilon & \rightarrow & {1\over T} \\
\mu & \rightarrow & {{M}\over{LT}} 
\end{eqnarray*}

Lets begin in the following way, I will use primed variables to
represent the non-dimensional form of the variable.   Lets think of solving for
the motion of the fluid in a box whose depth is $D$.   We need a characteristic
time scale for time.  To do this, our first assumption will be that we know
something about the characteristic velocity of the flow.

This leads to the following relations between the dimensional and
non-dimensional variables
\begin{eqnarray}
x & = & x^\prime D \\
u & = & u^\prime {V} \\
t & = & t^\prime {D\over V} \\
p & = & P_o + p^\prime {\mu V\over D}
\end{eqnarray}
Here I have pulled out a background pressure, $P_o$, because it is only
gradients in pressure that drive flow (see~\ref{eq:navier-stokes1}).   Keep in
mind that derivatives must be dimensionless too, so
\begin{equation} 
\nabla = \nabla^\prime {1\over D}
\end{equation}
Plugging these into  equation~\ref{eq:navier-stokes1}, we get
\begin{equation}     
{{\partial \vec u^\prime} \over {\partial t^\prime}}{{V}\over{D\over V}}
 + ( \vec u^\prime \cdot \nabla^\prime ) \vec u^\prime {V^2\over D} 
= -{1\over\rho}\nabla^\prime p^\prime {1\over D} {\mu V\over D}
+ {{\delta\rho}\over\rho}\vec g^\prime {{\mu V}\over D^2} +  
{\mu\over\rho} \left (\nabla^\prime \right )^2 \vec u^\prime {V\over D^2}.
\label{eq:navier-stokes2}
\end{equation}
Dividing through by ${{V^2}\over{D}}$
\begin{equation}      
{{\partial \vec u^\prime} \over {\partial t^\prime}}
 + ( \vec u^\prime \cdot \nabla^\prime ) \vec u^\prime   =
-\nabla^\prime p^\prime {\mu \over {\rho V D}} 
+ \delta\rho \vec g^\prime {\mu \over {\rho V D}} 
+ {\mu\over\rho} \left (\nabla^\prime \right )^2 \vec u^\prime {1\over {V D}}.
\label{eq:navier-stokes-nd}
\end{equation}
which we can rewrite as
\begin{equation}        
{{\partial \vec u^\prime} \over {\partial t^\prime}}
 + ( \vec u^\prime \cdot \nabla^\prime ) \vec u^\prime   
=  -\nabla^\prime p^\prime {1\over {\rm Re}}   
+ \delta\rho \vec g^\prime {1\over {\rm Re}}   
+  {1\over {\rm Re}} \left (\nabla^\prime \right )^2 \vec u^\prime.
\label{eq:navier-stokes-nd2}
\end{equation}
where
\begin{equation}
Re = {{\rho V D}\over\mu}
\end{equation}
is the Reynolds number.  There are several interesting things to think about
with the Reynolds number.   First examining the terms, you see that the
Reynolds number is dimensionless.   In fluid dynamics, there are a set,
actually a whole bunch, of dimensionless numbers that govern the dynamics of
fluid motions.   Next, notice that any combination of $\rho, V, D, {\rm and}
\mu$ which give the same Reynolds number will give the same answer to the
equations.   Hence, if we solve the dimensionless problem for a given Reynolds
number, we have a solution for a whole class of problems where $\rho, V, D, {\rm
and} \mu$ give the same Reynolds number.

Lets look at some specific examples of flows:

\begin{itemize}
\item Motion of the atmosphere:
\begin{eqnarray*}
\rho & = & 1.3~{{\rm kg}\over{\rm m^3}} \\ 
V    & = & 20.0~{{\rm km}\over{\rm hour}}    \\ 
D    & = & 10~{\rm km} \\
\mu  & = & 1.78 \time 10^{-5}~{\rm Pa~s} \\ 
Re   & = & {{1.3 \times 2.0 \times 10^4 \times 2.7 \times 10^{-4} \times 10^4 }
\over {1.78 \times 10^{-5}} } \\
     & = & 6 \times 10^{9}
\end{eqnarray*}

\item Water in a stream:
\begin{eqnarray*}
\rho & = & 10^3~{{\rm kg}\over{\rm m^3}} \\
V    & = & 1.0~{{\rm m}\over{\rm s}}    \\
D    & = & 10~{\rm m} \\
\mu  & = & 1.14\times 10^{-3}~{\rm Pa~s} \\
Re   & = & {{10^3 \times 1.0 \times 10.} \over {1.14 \times 10^{-3}} } \\
     & = & 10^7
\end{eqnarray*}

\item A surging glacier:
\begin{eqnarray*}
\rho & = & 900~{{\rm kg}\over{\rm m^3}} \\ 
V    & = & 10.0~{{\rm m}\over{\rm day}}    \\ 
D    & = & 100.0~{\rm m} \\
\mu  & = & 1.0\times 10^{6}~{\rm Pa~s} \\ 
Re   & = & {{900 \times 10.0 \times 1 \times 10^{-5} \times 100. } \over {1.0
\times 10^{10} }} \\
     & = & 9 \times 10^{-10}
\end{eqnarray*}

\item Motion of the Mantle Resulting from Plate Motions:
\begin{eqnarray*}
\rho & = & 3300~{{\rm kg}\over{\rm m^3}} \\  
V    & = & 10.0~{{\rm cm}\over{\rm yr}}    \\  
D    & = & 3000.0~{\rm km} \\
\mu  & = & 1.0\times 10^{21}~{\rm Pa~s} \\  
Re   & = & {{3300 \times 10.0 \times 3 \times 10^{-10} \times 3\times 10^6 }
\over {1.0 \times 10^{21} }} \\
     & = & 3 \times 10^{-20}
\end{eqnarray*}

\end{itemize}

Notice that when $Re \rightarrow 0$ equation~\ref{eq:navier-stokes-nd2} reduces
to {\em Stokes equation}
\begin{equation}        
0 =  -\nabla^\prime p^\prime  + \delta\rho \vec g^\prime  + \left
(\nabla^\prime \right )^2 \vec u^\prime.
\label{eq:stokes-nd}
\end{equation}

Considering the other end-member, when $Re \rightarrow \infty ,$
equation~\ref{eq:navier-stokes-nd2} almost reduces to Euler's equation.  If we
have chosen to scale the pressure by $\rho V^2$ the Reynolds number did not show
up in front of the pressure term, then equation~\ref{eq:navier-stokes-nd2} would
have reduced to Euler's equation.   This scaling ambiguity, suggests that the
pressure gradient term should always be left in the equations unless one has a
good physical mechanism to get rid of it.  

Luckily for Earth Scientists, many of the problems we want to solve are like
the problems I illustrated above--the Reynolds number is either very larger
or very small.  In either case,  one group of terms can be
neglected,  greatly simplifying the problem.   Note that we did not consider the
effect of rotation in our derivation of the equations.  The Earth is a
rotating body and all of the fluids moving around on and in Earth are subject
to the Coriolis force.  I don't want to go back and add that contribution, but
I will state that when you go through the non-dimensionalization, you get a
Reynolds number in front of that term as well.   Because the focus of this
course is on Stokes equation, or viscous flow, we will not mention the
effect of rotation any further.

There are several things to think about.  Stokes equation assumes
viscous stresses are balanced by buoyancy and pressure gradients.   There is no
inertia in the problem.  If you change the force, the fluid changes
instantaneously in response to the change in force.  There is also no time
dependence in Stokes equation.   Given a sent of boundary conditions for
velocity, Stokes equation gives a steady solution (time enters no where in the
equation).   Time can only enter in through the changing of the buoyancy terms
$\delta \rho$ or time-dependent boundary conditions.

At this point, you should be wondering what happens when we consider a case
where we don't have any idea what characteristic velocity we should use in the
scaling.  That is the case, for example, in convection problems.   We will
consider that case next class.

Now lets look at another way to scale time.   For this problem, we again
consider the motion of the fluid in a box whose depth is $D$.  Now, we
will assume that the fluid is convecting.  That means, the $\delta \rho$ term
in the buoyancy term is the result of temperature gradients in the fluid.  The
temperatures will be advected by the flow, making  the problem non-linear. 
This adds a number of complications, which for now we will not worry about. 
However, we don't have any idea what the velocity should be, so we can't use a
characteristic velocity as in the Reynolds number case above.  Another time
scale of interest will be related to the diffusion of heat. The parameter of
interest is the thermal diffusivity, $\kappa$.  By inspecting the units of
thermal diffusivity (${\rm m^2 / sec}$, we can get a time scale of $D^2/\kappa$.

This leads to the following relations between the dimensional and
non-dimensional variables
\begin{eqnarray} 
x & = & x^\prime D \\ 
t & = & t^\prime {D^2 \over \kappa} \\ 
u & = & u^\prime {\kappa \over D} \\ 
p & = & P_o + p^\prime {{\mu \kappa}\over D^2}
\end{eqnarray} 
Once again, I have pulled out a background pressure, $P_o$, because it
is only gradients in pressure that drive flow (see~\ref{eq:navier-stokes1}).  
 
Plugging these into equation~\ref{eq:navier-stokes1}, we get
\begin{equation}      
{{\partial \vec u^\prime} \over {\partial t^\prime}}
{{\kappa\over D}\over{D^2\over \kappa}}
+ (\vec u^\prime \cdot \nabla^\prime ) \vec u^\prime {{\kappa^2}\over D^3}  =
-{1\over\rho}\nabla^\prime p^\prime {1\over D} {{\mu\kappa}\over D^2} +
{{\delta\rho}\over\rho}\vec g^\prime {{\mu \kappa}\over D^3} + {\mu\over\rho}
\left (\nabla^\prime \right )^2 \vec u^\prime {\kappa\over D^3}.
\label{eq:navier-stokes4}
\end{equation} 
Dividing Equation~\ref{eq:navier-stokes4} by $\kappa^2\over D^3$ yields,
\begin{equation}       
{{\partial \vec u^\prime} \over {\partial t^\prime}}
 + ( \vec u^\prime \cdot \nabla^\prime ) \vec u^\prime   = 
-\nabla^\prime p^\prime {\mu \over {\rho \kappa}}  + 
\delta\rho \vec g^\prime {\mu \over {\rho \kappa}}  + 
{\mu\over\rho} \left (\nabla^\prime \right )^2 \vec u^\prime {1\over {kappa}}.
\label{eq:navier-stokes-ndpr}
\end{equation} 
We can simplify Equation~\ref{eq:navier-stokes-ndpr} by collecting terms
\begin{equation}         
{{\partial \vec u^\prime} \over {\partial t^\prime}} + ( \vec u^\prime \cdot
\nabla^\prime ) \vec u^\prime   =  -\nabla^\prime p^\prime {\rm Pr}   
+ \delta\rho \vec g^\prime {\rm Pr} + 
{\rm Pr} \left (\nabla^\prime \right )^2 \vec u^\prime.
\label{eq:navier-stokes-nd2pr}
\end{equation} 
where
\begin{equation} 
Pr = {\mu\over{\rho\kappa}} = {\nu \over \kappa}
\end{equation} 
is the Prandtl number.  Prandtl was the director of the Kaiser Wihelm Institute
for Fluid Research at the University of G\"ottingen.   His student, Tiejens,
wrote a classic text, ``Fundamentals of Hydro- and Aeromechanics'' (1934) which
is based on Prandtl's lecture notes.  It is an interesting and important
attempt to bridge the gap between the theoretical fluid dynamics world, where
approximations, even if not valid, were made to solve equations and the
experimental world which was a collection of disintegrated and
(seemingly) unrelated problems.  

Once again, the Prandtl number, like the Reynolds number, is dimensionless. Lets
think about the physical meaning of the Prandtl number.  One way you can think
of it is the ratio of the diffusion of momentum, the dynamic viscosity,
$\nu,$ to the diffusion of heat, $\kappa$, the thermal diffusivity.  Notice that
if I let the Prandlt number go to infinity we recover Stokes equation.  The
table below gives you the Prandtl number of some common fluids.

\begin{center}
\begin{tabular}{||c|c|c|c|c||} \hline
Fluid     & Density  & Viscosity & Thermal Diff-     & Prandtl number \\
          & kg/m$^3$ &  Pa s     & usivity (m$^2$/s) & (dimensionless)\\ \hline 
Air       &  1.3     & $10^{-5}$ & $2\times 10^{-5}$ &   0.72         \\
Water     & 1000.0   & $10^{-3}$ &$1.4\times 10^{-7}$&   8.10         \\
Olive Oil &  916.    &  0.1      &$9.2\times 10^{-8}$& 1,170          \\
Glycerine &  1260.   & 2.3       &$9.8\times 10^{-8}$& 18,880         \\
Mantle    & 3300.0   & $10^{21}$ &  $10^{-6}$        & $10^{30}$      \\ \hline
\end{tabular}
\end{center}

In practice, experimentalists usually consider a Prandtl number on the order of
1,000 to be large enough to use the infinite Prandtl number approximation.

\bigskip
To convince yourselves that you understand non-dimensionalization, attempt the
following problems.

\begin{quote} {\bf Problem 3} Return to Equation~\ref{eq:navier-stokes1} and use
the following relations between the dimensional and non-dimensional variables
\begin{eqnarray} 
x & = & x^\prime D \\ 
u & = & u^\prime {V} \\ 
t & = & t^\prime {D\over V} \\
p & = & P_o + p^\prime {\rho V^2} \\
\rho g & = & \rho {V^2 \over D}
\end{eqnarray}
What is the dimensionless form of the equation?  
\end{quote}

\begin{quote} {\bf Problem 4} You are asked to design a lab experiment to study
flow in the Wabash River.   Given that your lab space and budget will allow
you to build a tank that is 2 meters deep and you will use water, what
velocities will you need in the lab to reproduce the conditions on the Wabash
(where the water flows at between 1-10 meters/second).   Defend your answer.
\end{quote}

\vfill\eject

\subsection{Solution Strategies}

\subsubsection{One-Dimensional Channel Flow}

We now turn to some simple solutions for low Reynolds number, or infinite
Prandtl number, or viscous flow.  For the first problem, we will restrict our
attention to steady flow in one dimension.  The figure blow illustrates the
geometry of this flow.

\begin{figure}[h]
\begin{picture}(300,150)(0,0)
\put(12,25){\framebox(275,100){}}
\end{picture}
\caption{The 1-D Channel}
\end{figure}

We will assume the fluid is a constant viscosity, Newtonian fluid.  Thus we can
use, 
\begin{equation}
{\mu} \nabla^2 \vec u =  \nabla p + \delta\rho \vec g.
\end{equation}
We will assume that $\mu$ and $\rho$ are constant and that $ \delta\rho = 0$,
so there are no buoyancy forces driving this problem.  Remember that we can
write the equations for each component separately,
\begin{equation}
\mu \left (  {{\partial^2 u}\over{\partial x^2 }} +  {{\partial^2
u}\over{\partial z^2}} \right ) = {{\partial P}\over {\partial x}}.
\end{equation} 
and
\begin{equation}
\mu \left (  {{\partial^2 v}\over{\partial x^2 }} +  {{\partial^2
v}\over{\partial z^2}} \right ) = {{\partial P}\over {\partial z}}.
\end{equation}
Because the flow is only in the $x$ direction, we can ignore the second
equation ($v = 0$).  Notice that this also tells us that there are no
pressure gradients in the $z$ direction (${{\partial P}\over {\partial z}} =
0$).  In addition, we will assume ${{\partial u}\over {\partial x}} = 0$.  
However, we can not assume that  ${{\partial P}\over {\partial x}} =0$, and
could drive flow in the $u$ direction.

If you wish, you could balance the forces on the box, or you can return to our
equations above and get,
\begin{equation}
\label{eq:1d}
\mu  {{\partial^2 u}\over{\partial z^2}} = {{\partial P}\over {\partial x}}.
\end{equation}

If we assume ${{\partial P}\over {\partial x}}$ is independent of $z$ (which it
is, ask yourself why) we can integrate equation~\ref{eq:1d} twice and get
\begin{equation}
u = {1\over{2\mu}} {{\partial P}\over {\partial x}} z^2 + c_1 z + c_2
\end{equation}
where $c_1$ and $c_2$ arise because we have used indefinite integrals.

We can solve this simple problem for specific cases, by choosing boundary
conditions which allow us to solve for $c_1$ and $c_2$.   

For our first example, lets consider ${{\partial P}\over {\partial x}} =0$ and
a channel of height $z=D$.   Lets choose $u(0) = 0$ and $u(D) = U_0$.  From 
$u(0) = 0$ we immediately see that $c_2 = 0$.  From $u(D) = U_0$, we find that
$c_1 = {U_0 \over D}$.  Thus the solution is
\begin{equation} 
u(z) = z{U_0 \over D}.
\end{equation}
There are several things of interest here, first, the velocity is linear in $z$
and second, the velocity is independent of the viscosity $\mu$.   In this case,
the boundary conditions completely determine the solution (independent of the
material properties).   This is called {\em Couette flow}.
\begin{figure}[h]
\begin{picture}(300,150)(0,0)
\put(12,25){\framebox(275,100){}}
\end{picture}
\caption{Couette Flow}
\end{figure}

For our second example, lets consider the case where ${{\partial P}\over
{\partial x}} \neq 0$.   Lets eliminate the effect of non-zero velocity boundary
conditions, which we looked at in the example above and consider, $u(0) =
u(D) = 0$.   From $u(0) = 0$, we see that $c_2$ = 0. From $ u(D) = 0$, we find 
\begin{equation} 
0 = {1\over{2\mu}} {{\partial P}\over {\partial x}} D^2 + c_1 D
\end{equation}
or
\begin{equation}  
c_1 = {-1\over{2\mu}} {{\partial P}\over {\partial x}} D
\end{equation}
thus,
\begin{equation}  
u(z) = {1\over{2\mu}} {{\partial P}\over {\partial x}} \left ( z^2 - Dz \right
)
\end{equation}

Notice in this case that the solution is parabolic and by taking the
derivative of $u(z)$ and setting it equal to zero, we find the maximum
(minimum) is at $D/2$.   This is called {\em Poiseuille Flow.}
\begin{figure}[h]
\begin{picture}(300,150)(0,0)
\put(12,25){\framebox(275,100){}}
\end{picture}
\caption{Poiseuille Flow}
\end{figure}

A note here, because the equations of viscous flow are linear, we can
superimpose two solutions.  As long as we satisfy the boundary conditions
correctly, they the superposition of two solutions is also a solution to the
equation.   Because we now have a solution with a non-zero boundary condition
and a solution with a non-zero pressure gradient, we now have the tools to
solve all the possible combinations of 1-D problems. 

We can, and people have, used these equations to gain intuition about viscous
flow for Earth problems.   A simple solution which we looked at last class was
a simple model for plate motion and flow in the mantle.   This allowed us to
superimpose two solutions.   The general solution, when we have both a velocity
b.c. on one side of the channel and a pressure gradient is
\begin{equation}  
u = {1\over{2\mu}} {{\partial P}\over {\partial x}} 
\left ( z^2 - zD \right ) - z{U_0 \over D} + U_0
\end{equation}

(The subsection on the asthenosphere counter-flow still needs to be typeset. 
Sorry).


\subsubsection{Stream-Function Formulation}

There are a number of strategies for solving viscous flow problems in more than
one dimension, some of these can be done analytically (in special restricted
cases), some can simplify the equations so that standard computer software or
algorithms can be used.  The first strategy we will look at is the
stream-function technique. Lets return to the Stokes equation, with a constant
viscosity.
\begin{equation} 
\nabla^2 \vec u = \nabla p +{\bf Ra \,}T\hat k. \label{eq:const-stokes}
\end{equation} 
This, along with Equation~\ref{eq:incompress-cont} can be solved in 2-D
by using the stream-function.  If we define a function $\psi$ such that
\begin{eqnarray} 
u & = &-{{\partial \psi} \over {\partial y}} \\ 
v & = & {{\partial \psi} \over {\partial x}}  
\end{eqnarray} 
where $u$ and $v$ are the $x$ and $y$ components of $\vec u$. 
Notice that Equation~\ref{eq:incompress-cont} is satisfied by our choice of
$\psi $.  Substituting our definitions of $u$ and $v$ into
Equation~\ref{eq:const-stokes}, we get 
\begin{eqnarray}  
0 & = & {{\partial P} \over {\partial x}} + 
\mu \left ({{\partial^3 \psi} \over {\partial x^2 \partial y}} + {{\partial^3
\psi} \over {\partial y^3}} \right ) \label{eq:x1} \\ 
-{\bf Ra \,}T & = & -{{\partial P} \over {\partial y}} + 
\mu \left ({{\partial^3 \psi} \over {\partial x^3 }} +  {{\partial^3\psi} \over
{\partial x \partial y^2}} \right ) \label{eq:y1}
\end{eqnarray} 
Taking the derivative with respect to $y$ of Equation~\ref{eq:x1}
and the derivative with respect to $x$ of Equation~\ref{eq:y1} and adding them
we get a biharmonic equation for the stream function $\psi , $ 

\begin{equation} 
-{\bf Ra \,} {{\partial T}\over{\partial x}} =  
{{\partial^4 \psi} \over {\partial x^4}} + 
2 {{\partial^4 \psi} \over {\partial x^2\partial y^2}} + 
{{\partial^4 \psi} \over {\partial y^4}}. \label{eq:biharmonic}
\end{equation} 
In in a more compact notation, this can be written as
\begin{equation}  
-{\bf Ra \,} {{\partial T}\over{\partial x}} =   \nabla^4 \psi
\end{equation} 
The solution to the biharmonic equation is relatively straight-forward.
 
Before we do that, lets examine the meaning of the stream function a bit more.
We can write the stream function for the general one-dimensional channel flow
problem that we did last time.

The general solution, when we have a velocity b.c. on one side of the channel
and a pressure gradient is
\begin{equation} 
u = {1\over{2\mu}} {{\partial P}\over {\partial x}} \left ( z^2 - Zed \right ) -
z{U_0 \over D} + U_0
\end{equation}
using the definition of the stream function,
\begin{equation}  
u  =  -{{\partial \psi} \over {\partial y}} 
\end{equation}
and integrating to find $\phi$, we get
\begin{equation}  
\psi = {-1\over{2\mu}} {{\partial P}\over {\partial x}} \left (
{{z^3}\over 3} - {{z^2 D}\over 2} \right ) + z^2{U_0 \over {2D}} - U_0
\end{equation}

It is {\bf not} helpful to plot $\psi$ for the Couette and Poiseuille Flows.

Once again, to convince yourselves that you understand the 1-D problems, I
suggest the following two problems:

\begin{quote} {\bf Problem 5}  
Consider the steady, 1-D flow of a viscous fluid down an incline plane.  Assume
the flow occurs in a layer of thickness $h$, as shown in the figure below. 
Show that the velocity profile is given by:
\begin{equation}
u(y) = {{\rho g \sin \alpha}\over {2 \mu}} \left ( h^2 - y^2 \right )
\end{equation}
where $y$ is the coordinate measured perpendicular to the incline plane,
$\alpha$ is the angle of inclination, and $g$ is the acceleration due to
gravity.    

Hint:  First show that
\begin{equation}
{{\partial \tau}\over {\partial y}} = - \rho g \sin \alpha
\end{equation}
then apply the no slip condition at $y=h$ and $\tau = 0$ at $y=0$.  This is the
{\em free-slip} condition.
\end{quote}
\begin{figure}[h]
\begin{picture}(300,150)(0,0)
\put(12,25){\framebox(275,100){}}
\end{picture}
\caption{Inclined Channel For Problem 5}
\end{figure}

\begin{quote} {\bf Problem 6}  Suppose you have a moving plate sitting on top
of a fluid that has two layers as pictured.  Our solution for Couette flow
only told us how to solve for a single viscosity fluid, but by matching the
stresses just above and below the viscosity interface, you can solve this
problem.  Using the values in the picture, solve for the steady velocity
profile.  
\end{quote}
\begin{figure}[h]
\begin{picture}(300,150)(0,0)
\put(12,25){\framebox(275,100){}}
\end{picture}
\caption{Two Layer Channel For Problem 6}
\end{figure}

\subsection{Stream-Function Examples}

\subsubsection{Post Glacial Rebound}

The growth and melting of large ice sheets occurs on a time scale so fast
compared to the mantle that loading and unloading the ice sheet can change flow
in the mantle.   During the last great ice age, Scandinavia and North America
were covered with thick sheets of ice.  This caused a considerable subsidence
of the surface.  When the ice sheets melting, the surface began to return to
its normal elevation (rebound).  The rate of rebound has been dated by changed
in elevation of ancient shore lines.

To study this problem, we will use the flow of a semi-infinite, viscous, half
space.  We will consider $y > 0$ subject to a periodic initial surface
displacement given by
\begin{equation}
w_m = w_{m0} \cos {{2 \pi x}\over{\lambda}}
\end{equation}
where $\lambda$ is the wavelength of the initial displacement and $w_m \ll
\lambda$.   The displacement leads to a hydrostatic pressure gradient which
drives flow, attempting to restore the surface to the undeformed state ($w =
0$).   We will use the stream-function formulation, solving the bi-harmonic
equation, to determine the viscous flow.   Because the initial displacement
varies in $x$ like $\cos {{2 \pi x}\over{\lambda}}$ it is reasonable to assume
that the stream function will also vary like a sin or cos in $x$.  Recall that
$\psi$ is an integral of velocity, so it is reasonable to expect that $\psi
\propto \sin {{2 \pi x}\over{\lambda}}$.  We will assume this form for the
solution and assume that the depth variation can be separated from the
horizontal variation, i.e.,
\begin{equation} 
\psi = \sin {{2 \pi x}\over{\lambda}} Y(y)
\end{equation}
where $Y$ is the function we will solve for.   If we substitute this form into
the biharmonic equation~\ref{eq:biharmonic}, we get
\begin{equation}
\frac{d^4 Y}{dy^4} - 2 \left [ {{2 \pi}\over{\lambda}} \right ]^2 
\frac{d^2 Y}{dy^2} + \left [ {{2 \pi}\over{\lambda}} \right ]^4 Y = 0.
\label{eq:forth-order-eq}
\end{equation}
Solutions of this equation will be of the form 
\begin{equation}
Y(y) \propto \exp (my).
\end{equation}
If this form is substituted into the above equation for $Y(y)$, we find that
$m$ is the solution of
\begin{equation}
m^4 - 2 \left [ {{2 \pi}\over{\lambda}} \right ]^2 m^2 + \left [ {{2
\pi}\over{\lambda}} \right ]^4 = \left ( m^2 -\left [ {{2 \pi}\over{\lambda}}
\right ]^2 \right )^2 = 0
\end{equation}
or $m = \pm {{2 \pi}\over{\lambda}}$.   This provides for two of the four
possible solutions for $Y$.   By direct substitution, you can verify that
\begin{eqnarray}
Y(y) & \propto & y \exp \left [ {{2 \pi y}\over{\lambda}} \right ] \\
Y(y) & \propto & y \exp \left [ -{{2 \pi y}\over{\lambda}} \right ]
\end{eqnarray}
are also solutions to~\ref{eq:forth-order-eq}. So the general solution can be
written as
\begin{equation}
\psi = \sin \left (\frac{2 \pi x}{\lambda} \right ) \left [ A e^{\frac{-2 \pi
y}{\lambda}} + By e^{\frac{-2 \pi y}{\lambda}} + C e^{\frac{2 \pi y}{\lambda}}
+ Dy e^{\frac{2 \pi y}{\lambda}} \right ]
\end{equation}
with the four constants $A, B, C, D$ to be determined by the boundary
conditions.  We first use the fact that we want finite velocities as the depth
of the half-space goes to infinity.  Therefore, $C$ and $D$ must be zero.  So
the formula for the stream function reduces to
\begin{equation}
\psi = \sin \left (\frac{2 \pi x}{\lambda} \right ) 
e^{\frac{-2 \pi y}{\lambda}} \left [ A + By \right ]
\end{equation}
We can now differentiate $\psi$ to get the velocity components
\begin{eqnarray}
u & = & \sin \left (\frac{2 \pi x}{\lambda} \right )  
e^{\frac{-2\pi y}{\lambda}}\left [\frac{2\pi}{\lambda}(A + By)-B\right ] \\
v & = & \cos \left (\frac{2 \pi x}{\lambda} \right ) \frac{2 \pi}{\lambda}
e^{\frac{-2 \pi y}{\lambda}} \left [ A + By \right ]
\end{eqnarray}
Because of the extreme different in properties between the mantle and the
atmosphere (here I am assuming that the lithosphere/crust is the same as the
mantle, which for the wavelengths we are talking about is not too bad an
assumption) we force the horizontal velocity of the flow to be zero at $y =
w$.  Now we use the fact that $w \ll \lambda$ so that we can apply this
condition to $y=0$.  By setting $u=0$ at $y=0$ we find that $B = \frac{2\pi
A}{\lambda}$ which means
\begin{eqnarray} 
\psi & = & A \sin \left (\frac{2 \pi x}{\lambda} \right )  e^{\frac{-2 \pi
y}{\lambda}} \left [ 1 + \frac{2 \pi y}{\lambda} \right ] \\
u & = & A \left ( \frac{2\pi}{\lambda} \right )^2
\sin \left (\frac{2 \pi x}{\lambda} \right ) y e^{\frac{-2\pi y}{\lambda}} \\ 
v & = & A \left ( \frac{2 \pi}{\lambda} \right )
\cos \left (\frac{2 \pi x}{\lambda} \right ) e^{\frac{-2 \pi y}{\lambda}} 
\left [ 1 + \frac{2 \pi y}{\lambda} \right ]
\end{eqnarray}
In order to find the constant $A$, we need to equate the hydrostatic pressure
head associated with the topography $w$ to the normal stress at the upper
boundary.   They hydrostatic pressure is $-\rho g w$ and the normal stress is
$p - 2\mu \frac{\partial v}{\partial y}$.   Once again because the
displacements are small, this can be done at $y=0$.   First, we must calculate
the pressure at $y=0$, which we can do by putting our expressions for $u$ and
$v$ back into the viscous flow equation (not the biharmonic).   Doing this, we
obtain
\begin{equation}
\frac{\partial p}{\partial x} = -2 \mu A \left ( \frac{2 \pi}{\lambda} \right
)^3 \sin \frac{2 \pi x}{\lambda}
\end{equation}
Integrating this we get
\begin{equation}
p = 2 \mu A \left ( \frac{2 \pi}{\lambda} \right
)^2 \cos \frac{2 \pi x}{\lambda}
\end{equation}
on $y=0$.  We also need $\frac{\partial v}{\partial y}$ at $y=0$ which
fortunately is $0$.  Thus
\begin{eqnarray}
-\rho g w & = & p - 2\mu \frac{\partial v}{\partial y}\\
w_{y=0} & = & \frac{-2 \mu A}{\rho g} \left ( \frac{2 \pi}{\lambda} \right )^2
\cos\frac{2 \pi x}{\lambda}
\end{eqnarray}

The key step is to notice that the time derivative of displacement $w$ is the
vertical velocity (at $y=w$).  But once again we use the fact that $w \ll
\lambda$ so that we can assume this at $y=0$.   At $y-0$ we have
\begin{equation}
v_{y=0} = A \frac{2 \pi}{\lambda}
\cos \left (\frac{2 \pi x}{\lambda} \right ) 
\end{equation}
so
\begin{equation} 
\frac{d w}{dt}_{y=0} = A \frac{2 \pi}{\lambda}
\cos \left (\frac{2 \pi x}{\lambda} \right ) 
\end{equation}
or
\begin{equation} 
\frac{d w}{dt}_{y=0} = -w \frac{\lambda g \rho}{4 \pi \mu}
\end{equation} 
we can integrate this to get
\begin{equation}
w(t) = w_{m0} \exp \left ( \frac{\lambda g \rho t}{4 \pi \mu} \right )
\end{equation}
thus the surface rebounds exponentially with time and fluid flows from regions
of elevated topography to regions of low topography.  We can think of the
grouping of constants $ \frac{4 \pi \mu}{\lambda g \rho} $ as a characteristic
time (i.e., the time it takes the topography to decay by 1/e).  Notice that
this grouping has units of time.   If we know the density (we do approximately)
and we know $g$ and the initial wavelength, $\lambda$, which we know pretty
well, then we can estimate the viscosity.

This exercise was first discussed by Norman Haskell, in 1935.  Haskell came up
with a value (in todays units) of $10^{21}$ Pa s which is sometimes called the
Haskell value. 

\medskip
\noindent
Haskell, N.A., The motion of a viscous fluid under a surface load, I,
{\em Physics,} 6, 265-269, 1935. 

\medskip
\noindent
Haskell, N.A., The motion of a viscous fluid under a surface load, II, 
{\em Physics,} 7, 56-61, 1936.

It is important to note that this is an average value, or effective value if
the Earth were a homogeneous fluid (which it is not).  Subsequent work has
focused on layered, then radial models (the spherical nature is not a
significant factor for all the additional work it contributes).  Layered
half-spaces are a fairly easy step from the discussion here.  It is also
straight-forward to consider an elastic lithosphere over a viscous half-space.
A detailed discussion of these effects can be found in

\medskip
\noindent
Cathles, L.M., {\em The Viscosity of the Earth's Mantle,} 386 pp., Princeton
University Press, Princeton, NJ, 1975.

\medskip

Current research focuses on laterally varying material properties.   The
methods that we will go on to discuss are needed to solve that level of problem.

\subsubsection{Corner Flow and the Angle of Subduction}

One of the most curious problems in mantle convection is that slabs do not
descend vertically.   All models of convection of a viscous fluid produce
symmetric, two-sided downwelling unless something is added to break the
symmetry, like a dipping fault or some difference in the properties of the two
thermal boundary layers.  Some data

\begin{center}
\begin{tabular}{||l|r||} \hline 
Arc & Dip Angle\\ \hline
Central Chile  &  5$^\circ$ \\
Northern Chile & 30$^\circ$ \\
Southern Chile & 30$^\circ$ \\
Honshu         & 30$^\circ$ \\
Izu-Bonin      & 60$^\circ$ \\
Java           & 70$^\circ$ \\
New Hebrides   & 70$^\circ$ \\
Ryukyu         & 45$^\circ$ \\
West Indies    & 50$^\circ$ \\ \hline
\end{tabular}
\end{center}

One explanation for why the slab descends at an angle other than $90^\circ$ is
that pressure forces due to the induced flow pull the slab up and
counter-balance the buoyancy forces.  We can investigate this with a stream
function solution.

As unlikely as it may seem, it is possible to write $\psi$ in the form
\begin{equation}
\psi ~=~ A\,x + B\,y + \left( {\rm C}\,x + {\rm D}\,y \right) \,
   \arctan ({y\over x})
\end{equation}
where $A, B, C,$ and  $D$ are constants to be determined by the boundary
conditions.   Because of the geometry of the descending slab, there are two
distinct stream functions, i.e., two sets of constants to be solved for in this
problem: one above and one below the slab.
It is helpful to recall that
\begin{eqnarray}
\frac{\partial}{\partial y} \arctan \frac{y}{x} & = & \frac{x}{x^2+y^2} \\
\frac{\partial}{\partial x} \arctan \frac{y}{x} & = & \frac{-y}{x^2+y^2} 
\end{eqnarray}
Differentiating the stream function to find the velocities, we find that
\begin{eqnarray}
u(x,y) & = & -\frac{\partial\psi}{\partial y} ~=~ -B - {{{\rm C}\,x + {\rm
D}\,y}\over{x\,\left( 1 + {{{y^2}}\over {{x^2}}} \right) }} - 
  {\rm D}\,\arctan ({y\over x}) \\
v(x,y) & = & \frac{\partial\psi}{\partial x} ~=~ A - {{y\,\left( {\rm C}\,x +
{\rm D}\,y \right) }\over{{x^2}\,\left( 1 + {{{y^2}}\over {{x^2}}} \right) }} + 
  {\rm C}\,\arctan ({y\over x}).
\end{eqnarray}

First, notice that this will satisfy continuity, because
\begin{equation}
\frac{\partial u}{\partial x} ~=~ {{-2\,{y^2}\,\left( {\rm C}\,x + {\rm D}\,y
\right) }\over 
    {{x^4}\,{{\left( 1 + {{{y^2}}\over {{x^2}}} \right) }^
        2}}} - {{{\rm C}}\over 
    {x\,\left( 1 + {{{y^2}}\over {{x^2}}} \right) }} + 
  {{{\rm D}\,y}\over 
    {{x^2}\,\left( 1 + {{{y^2}}\over {{x^2}}} \right) }} + 
  {{{\rm C}\,x + {\rm D}\,y}\over 
    {{x^2}\,\left( 1 + {{{y^2}}\over {{x^2}}} \right) }} \label{eq:ux}
\end{equation}
and
\begin{equation}
\frac{\partial v}{\partial y} ~=~{{2\,{y^2}\,\left( {\rm C}\,x + {\rm D}\,y
\right) }\over 
    {{x^4}\,{{\left( 1 + {{{y^2}}\over {{x^2}}} \right) }^
        2}}} + {{{\rm C}}\over 
    {x\,\left( 1 + {{{y^2}}\over {{x^2}}} \right) }} - 
  {{{\rm D}\,y}\over 
    {{x^2}\,\left( 1 + {{{y^2}}\over {{x^2}}} \right) }} - 
  {{{\rm C}\,x + {\rm D}\,y}\over 
    {{x^2}\,\left( 1 + {{{y^2}}\over {{x^2}}} \right) }} \label{eq:vy}
\end{equation}
By inspection ~\ref{eq:ux} plus \ref{eq:vy} sum to zero.


As in the last problem, the pressure can be found by substituting the solution
for $u$ back into the equation of motion and solving for $\frac{\partial
p}{\partial x}$ and integrating in $x$.   To see this notice that
\begin{displaymath}
\frac{\partial^2 u}{\partial x^2} ~=~ {{-8\,{y^4}\,\left( {\rm C}\,x + {\rm
D}\,y \right) }\over 
    {{x^7}\,{{\left( 1 + {{{y^2}}\over {{x^2}}} \right) }^
        3}}} - {{4\,{\rm C}\,{y^2}}\over 
    {{x^4}\,{{\left( 1 + {{{y^2}}\over {{x^2}}} \right) }^
        2}}} + {{2\,{\rm D}\,{y^3}}\over 
    {{x^5}\,{{\left( 1 + {{{y^2}}\over {{x^2}}} \right) }^
        2}}} + 
\end{displaymath}
\begin{equation}
{{10\,{y^2}\,
      \left( {\rm C}\,x + {\rm D}\,y \right) }\over 
    {{x^5}\,{{\left( 1 + {{{y^2}}\over {{x^2}}} \right) }^
        2}}} + {{2\,{\rm C}}\over 
    {{x^2}\,\left( 1 + {{{y^2}}\over {{x^2}}} \right) }} - 
  {{2\,{\rm D}\,y}\over 
    {{x^3}\,\left( 1 + {{{y^2}}\over {{x^2}}} \right) }} - 
  {{2\,\left( {\rm C}\,x + {\rm D}\,y \right) }\over 
    {{x^3}\,\left( 1 + {{{y^2}}\over {{x^2}}} \right) }}
\end{equation}
and
\begin{equation}
\frac{\partial^2 u}{\partial y^2} ~=~{{-8\,{y^2}\,\left( {\rm C}\,x + {\rm D}\,y
\right) }\over 
    {{x^5}\,{{\left( 1 + {{{y^2}}\over {{x^2}}} \right) }^
        3}}} + {{6\,{\rm D}\,y}\over 
    {{x^3}\,{{\left( 1 + {{{y^2}}\over {{x^2}}} \right) }^
        2}}} + {{2\,\left( {\rm C}\,x + {\rm D}\,y \right)}
     \over {{x^3}\,{{\left( 1 + {{{y^2}} \over {{x^2}}} \right
           ) }^2}}}
\end{equation}
thus,
\begin{eqnarray}
\frac{\partial p}{\partial x} & = &\mu \left ( \frac{\partial^2 u}{\partial x^2}
+ \frac{\partial^2 u}{\partial y^2} \right ) =  \nonumber \\
&& \mu \,\left( {{-8\,{y^2}\,
        \left( {\rm C}\,x + {\rm D}\,y \right) }\over 
      {{x^5}\,{{\left( 1 + {{{y^2}}\over {{x^2}}} \right) }^
          3}}} - {{8\,{y^4}\,
        \left( {\rm C}\,x + {\rm D}\,y \right) }\over 
      {{x^7}\,{{\left( 1 + {{{y^2}}\over {{x^2}}} \right) }^
          3}}} + {{6\,{\rm D}\,y}\over 
      {{x^3}\,{{\left( 1 + {{{y^2}}\over {{x^2}}} \right) }^
          2}}} - \right. \nonumber \\
 &&   {{4\,{\rm C}\,{y^2}}\over 
      {{x^4}\,{{\left( 1 + {{{y^2}}\over {{x^2}}} \right) }^
          2}}} + {{2\,{\rm D}\,{y^3}}\over 
      {{x^5}\,{{\left( 1 + {{{y^2}}\over {{x^2}}} \right) }^
          2}}} + {{2\,\left( {\rm C}\,x + {\rm D}\,y \right )
        }\over 
      {{x^3}\,{{\left( 1 + {{{y^2}}\over {{x^2}}} \right ) }^
          2}}} + {{10\,{y^2}\,
        \left( {\rm C}\,x + {\rm D}\,y \right) }\over 
      {{x^5}\,{{\left( 1 + {{{y^2}}\over {{x^2}}} \right) }^
          2}}} +  \nonumber \\
 &&   \left. {{2\,{\rm C}}\over 
      {{x^2}\,\left( 1 + {{{y^2}}\over {{x^2}}} \right) }} -
      {{2\,{\rm D}\,y}\over 
      {{x^3}\,\left( 1 + {{{y^2}}\over {{x^2}}} \right) }} -
      {{2\,\left( {\rm C}\,x + {\rm D}\,y \right) }\over 
      {{x^3}\,\left( 1 + {{{y^2}}\over {{x^2}}} \right) }} \right )
\label{eq:harry-mess}
\end{eqnarray}
and finally integrating~\ref{eq:harry-mess} with respect to $x$ gives
\begin{equation}
p(x,y) ~=~ {{-2\,\mu \,\left( {\rm C}\,x + {\rm D}\,y \right) }\over 
   {{x^2} + {y^2}}}
\end{equation}
Now we are ready to apply boundary conditions to the problem.

The general problem leads to quite complicated integrals.  We can look at some
specific cases.  A dip of $\pi/4$ is a good choice.   We will first look at the
solution in the arc corner. We have the following boundary conditions
\begin{eqnarray}
u=v  & = & 0 ~~~~~~~~~~~~~~~~~~~~~on~y=0,~x>0~or~\arctan \frac{y}{x}~=~0 \\
u=v  & = & {U\sqrt{2}\over 2}~~~~~~~~~~~~~~~on~y=x~or~\arctan
\frac{y}{x}~=~\frac{\pi}{4} 
\end{eqnarray}
Substituting $y=0$ into $u(x,y) = 0$, we find
\begin{equation}
-B -C = 0. \label{eq:bc1}
\end{equation}
Substituting $y=0$ into $v(x,y) = 0$, we find
\begin{equation}
A=0. \label{eq:bc2}
\end{equation}
Substituting $y=x (\arctan\frac{y}{x} = \pi/4)$, into $u(x,y) = U\sqrt{2}/2$, we
find
\begin{equation}
-B - \frac{C}{2} -\frac{D}{2} -\frac{D\pi}{4} =  U\sqrt{2}/2. \label{eq:bc3}
\end{equation}
Substituting $y=x (\arctan\frac{y}{x} = \pi/4)$, into $v(x,y) = U\sqrt{2}/2$, we
find
\begin{equation} 
A - \frac{C}{2} -\frac{D}{2} + \frac{C\pi}{4} =  U\sqrt{2}/2.
\label{eq:bc4}
\end{equation}
Simultaneously solving~\ref{eq:bc1}, \ref{eq:bc2}, \ref{eq:bc3}, and
\ref{eq:bc4}, leads to the following expressions for
$C$ and $D$,
\begin{eqnarray}
C & = & \frac{-\pi U\sqrt{2}}{2(2-\pi^2/4)} \\
D & = & \frac{-U\sqrt{2}(2-\pi/2)}{(2-\pi^2/4)}
\end{eqnarray}
thus the pressure in the arc corner is
\begin{equation}
p = \frac{\mu U \sqrt{2} \left [ \pi x + (4-\pi)y \right]}{(2-\pi^2/4)(x^2+y^2)}
\end{equation}
we can evaluate this expression using the fact that $x=y=r\sqrt{2}/2$ where $r$
is the length along the slab.   Thus we find that 
\begin{equation}
p = \frac{4\mu U}{(2-\pi^2/4)r} = \frac{-8\mu U}{r}
\end{equation}
The negative pressure means the back-arc region flow tends to lift the slab
against the force of gravity.   Notice that the pressure varies as $1/r$ along
the slab, with a singularity at the trench.  The total lifting torque on the
slab will be $r\times P$ which is constant.

For the ocean side of the arc, the boundary conditions are
\begin{eqnarray} 
u=U, v  & = & 0 ~~~~~~~~~~~~~~~~~~~on~y=0,~x<0~or~\arctan \frac{y}{x}~=~\pi \\ 
u=v  & = & {U\sqrt{2}\over 2}~~~~~~~~~~~~~on~y=x~or~\arctan
\frac{y}{x}~=~\frac{\pi}{4} 
\end{eqnarray}
Substituting $y=0, (\arctan \frac{y}{x}= \pi )$ into $u(x,y) = U $ we get
\begin{equation}
-B - C - D \pi = U. \label{eq:bc5}
\end{equation}
Substituting $y=0, (\arctan \frac{y}{x}= \pi )$ into $v(x,y) = 0 $ we get
\begin{equation} 
A + C \pi = 0. \label{eq:bc6}
\end{equation}
Substituting $y=x, (\arctan \frac{y}{x}= \pi/4 )$ into $u(x,y) = U\sqrt{2}/2 $ we
get
\begin{equation} 
-B - \frac{C}{2} -\frac{D}{2} -\frac{D\pi}{4} =  U\sqrt{2}/2. \label{eq:bc7}
\end{equation} 
Substituting $y=x, (\arctan \frac{y}{x}= \pi/4 )$ into $v(x,y) = U\sqrt{2}/2 $
we get
\begin{equation} 
A - \frac{C}{2} -\frac{D}{2} + \frac{C\pi}{4} =  U\sqrt{2}/2. \label{eq:bc8}
\end{equation}
Solving \ref{eq:bc5}, \ref{eq:bc6}, \ref{eq:bc7}, and \ref{eq:bc8} for $C$ and
$D$, we get
\begin{eqnarray}
C & = & \frac{U}{(9\pi^2/4-2)} \left [ 2 - \frac{\sqrt{2}}{(1-3\pi/2)} \left (
\frac{3\pi}{2} + \frac{9\pi^2}{4} \right ) \right ] \\
D & = & \frac{U}{(9\pi^2/4-2)} \left [\sqrt{2} \left (2+\frac{3\pi}{2} \right )
-2 \left ( 1+\frac{3\pi}{2} \right ) \right ]
\end{eqnarray}
The pressure on the bottom of the slab is given by
\begin{equation}
p= \frac{\mu U}{r} \left ( \frac{3\pi\sqrt{2} -4}{9\pi^2/4-2} \right ) =
\frac{0.5 \mu U}{r}.
\end{equation}
Thinking about the geometry and the signs of the pressures, this flow also
exerts a lifting torque on the slab, it is about a factor of 20 smaller than
the suction torque.   However, because constant viscosity convection never
produces this kind of solution, we are lead to suspect that the downward force
of gravity far outweighs the pressure torques.   It would be an interesting
exercise to see if the problem with both sides coming together at the same
velocity produced torques that balanced at some angle...think about this.

\end{document}